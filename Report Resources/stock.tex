\section{Les diagrammes commutatifs}
%\mypart{Les diagrammes commutatifs}

%%%%%%%%%%%%%%%%%%%%%%%%%%%%%%%%%%%%%%%%%%%%%%%%%%%%%%%%%%%%%%%%%%%
\begin{frame}[fragile]
\frametitle{L'approche AMS-\LaTeX\ 1/3}

\begin{itemize}

\item À l'aide du package \texttt{amscd} on peut composer des diagrammes commutatifs simples ;

\item on écrit, dans un environnement \texttt{CD}, lui même en mode mathématique, les sommets du diagramme et les flèches (les flèches verticales faisant une ligne à part) ;

\item pour obtenir les flèches on écrira \verb=@>>>=, \verb=@<<<=, \verb=@VVV= et \verb=@AAA=, ou alors \verb=@|= et \verb:@=: pour les égalités horizontale et verticale, et \verb=@.= pour la flèche vide ;


\end{itemize}

\end{frame}

%%%%%%%%%%%%%%%%%%%%%%%%%%%%%%%%%%%%%%%%%%%%%%%%%%%%%%%%%%%%%%%%%%%
\begin{frame}[fragile]
\frametitle{L'approche AMS-\LaTeX\ 2/3}

$$\begin{CD}
A @<<< B @>>> C @>>> D\\
@VVV   @AAA   @VVV   @AAA\\
E @>>> F @>>> G @>>> H
\end{CD}$$

\begin{verbatim}
$$\begin{CD}
A @<<< B @>>> C @>>> D\\
@VVV   @AAA   @VVV   @AAA\\
E @>>> F @>>> G @>>> H
\end{CD}$$
\end{verbatim}

\end{frame}

%%%%%%%%%%%%%%%%%%%%%%%%%%%%%%%%%%%%%%%%%%%%%%%%%%%%%%%%%%%%%%%%%%%
\begin{frame}[fragile]
\frametitle{L'approche AMS-\LaTeX\ 3/3}

\begin{itemize}

\item On peut enrichir les diagrammes commutatifs \og à la AMS-\LaTeX\fg{} en ajoutant des variables au-dessus ou au-dessous des flèches horizontales : 

\texttt{@>\{\textit{au-dessus}\}>\{\textit{au-dessous}\}>} ;

\item de même, on peut ajouter des variables à gauche ou à droite des flèches verticales :

\texttt{@V\{\textit{à gauche}\}V\{\textit{à droite}\}V} ;

\item ni les égalités, ni la flèche vide ne peut être lettrée.

\end{itemize}

\end{frame}

%%%%%%%%%%%%%%%%%%%%%%%%%%%%%%%%%%%%%%%%%%%%%%%%%%%%%%%%%%%%%%%%%%%
\begin{frame}[fragile]
\frametitle{L'approche XY-pic 1/6}

\begin{itemize}

\item On charge le package \texttt{xy} avec l'option \texttt{all} (et  \texttt{ps} et \texttt{dvips} si on n'est pas sous \emph{pdf\TeX}) ;

\item un diagramme commutatif est écrit dans un élément \verb=\xymatrix= (lui-même dans un environnement mathématique) ;

\item comme dans \texttt{CD} on écrit les sommets, mais les flèches se mettent juste après les sommets \emph{dont elles partent}. La syntaxe est \verb=\ar[.]= où \texttt{.} est \texttt{u} (vers le haut), \texttt{d} (le bas), \texttt{r} (droite), \texttt{l} (gauche) :

$$\xymatrix{
A \ar[d] & B \ar[l] \ar[r] & C \ar[r] \ar[d] & D \\
E \ar[r] & F \ar[u] \ar[r] & G \ar[r] & H \ar[u]
}$$

\begin{verbatim}
$$\xymatrix{
A \ar[d] & B \ar[l] \ar[r] & C \ar[r] \ar[d] & D \\
E \ar[r] & F \ar[u] \ar[r] & G \ar[r] & H \ar[u]}$$
\end{verbatim}

\end{itemize}

\end{frame}

%%%%%%%%%%%%%%%%%%%%%%%%%%%%%%%%%%%%%%%%%%%%%%%%%%%%%%%%%%%%%%%%%%%
\begin{frame}[fragile]
\frametitle{L'approche XY-pic 2/6}

\begin{itemize}
\item Pour ajouter des labels aux flèches, on utilise \verb=^= (au-dessus), \verb=|= (au milieu) et \verb=_= (au-dessous) après la commande \verb=\ar[.]= : $\xymatrix{X \ar[r]^{f}_{g} & Y}$ obtenu par \verb=$\xymatrix{X \ar[r]^{f}_{g} Y}$= ;

\item pour centrer une entrée sur la flèche, écrire un tiret après le chapeau (resp. le souligné ou la barre verticale) : \verb=^-=, \verb=_-=, \verb=|-= ;

\item les flèches diagonales s'obtiennent en mettant dans l'argument optionnel de \verb=\ar= autant de lettres \texttt{u}, \texttt{d}, \texttt{r} et \texttt{l} qu'il y a des pas dans la direction correspondante :

$$\xymatrix{
A \ar[r] \ar[d] \ar[dr] \ar[drr] \ar[drrr] \ar[drrrr] \ar[drrrrr] &
B \ar[r] & C \ar[r] & D \ar[r] & E \ar[r] & F \ar[d]\\
A' \ar[r] & B' \ar[r] & C' \ar[r] & D' \ar[r] & E' \ar[r] & F'
}$$


\end{itemize}

\end{frame}

%%%%%%%%%%%%%%%%%%%%%%%%%%%%%%%%%%%%%%%%%%%%%%%%%%%%%%%%%%%%%%%%%%%
\begin{frame}[fragile]
\frametitle{L'approche XY-pic 3/6}

\begin{verbatim}
$$\xymatrix{
A \ar[r] \ar[d] \ar[dr] \ar[drr] 
\ar[drrr] \ar[drrrr] \ar[drrrrr] &
B \ar[r] & C \ar[r] & D \ar[r] & E \ar[r] & F \ar[d]\\
A' \ar[r] & B' \ar[r] & C' \ar[r] 
& D' \ar[r] & E' \ar[r] & F'
}$$
\end{verbatim}

\end{frame}

%%%%%%%%%%%%%%%%%%%%%%%%%%%%%%%%%%%%%%%%%%%%%%%%%%%%%%%%%%%%%%%%%%%
\begin{frame}[fragile]
\frametitle{L'approche XY-pic 4/6}

\begin{itemize}
\item Les flèches deviennent curvilignes quand on ajoute \verb=@/^/= ou \verb=@/_/= :

$$\xymatrix{
A \ar[r] \ar[dd] \ar@/^/[ddr] \ar@/^/[ddrr] 
\ar@/^/[ddrrr] \ar@/^/[ddrrrr] \ar@/^/[ddrrrrr] &
B \ar[r] & C \ar[r] & D \ar[r] & E \ar[r] & F \ar[dd]\\
{} & {} & {} & {}& {}\\
A' \ar[r] & B' \ar[r] & C' \ar[r] & D' \ar[r] & E' \ar[r] & F'
}$$

\begin{verbatim}
$$\xymatrix{A \ar[r] \ar[dd] \ar@/^/[ddr] 
\ar@/^/[ddrr] \ar@/^/[ddrrr] \ar@/^/[ddrrrr] 
\ar@/^/[ddrrrrr] & B \ar[r] & C \ar[r] & 
D \ar[r] & E \ar[r] & F \ar[dd]\\
{} & {} & {} & {}& {}\\ A' \ar[r] & B' \ar[r] 
& C' \ar[r] & D' \ar[r] & E' \ar[r] & F'}$$
\end{verbatim}

\end{itemize}

\end{frame}

%%%%%%%%%%%%%%%%%%%%%%%%%%%%%%%%%%%%%%%%%%%%%%%%%%%%%%%%%%%%%%%%%%%
\begin{frame}[fragile]
\frametitle{L'approche XY-pic 5/6}

\begin{itemize}
\item Les flèches de différents styles s'obtiennent par \verb:@{=>}: (double flèche), \verb:@{.>}: (flèche pointillée), \verb=@{:>}= (flèche pointillée double), \verb=@{~>}= (flèche ondulée), \verb=@{-->}= (flèche en petits traits), \verb=@{}= (flèche vide) ;

$$\xymatrix{A \ar[r] \ar@{}[dr]|{\circlearrowleft} \ar[d] 
& B \ar@{.>}[r] \ar[d] & C \ar@{.>}[d]\\
D \ar[r] & E \ar@{~>}[r] & F}$$

\begin{verbatim}
$$\xymatrix{A \ar[r] \ar@{}[dr]|{\circlearrowleft} 
\ar[d] & B \ar@{.>}[r] \ar[d] & C \ar@{.>}[d]\\
D \ar[r] & E \ar@{~>}[r] & F}$$
\end{verbatim}

\end{itemize}

\end{frame}

%%%%%%%%%%%%%%%%%%%%%%%%%%%%%%%%%%%%%%%%%%%%%%%%%%%%%%%%%%%%%%%%%%%
\begin{frame}[fragile]
\frametitle{L'approche XY-pic 6/6}

\begin{itemize}
\item On peut déplacer les flèches transversalement à l'aide de \verb=@<1ex>= :

$$\xymatrix{A \ar@<.5ex>@/^/[r] \ar@<-.5ex>@/^/[r] \ar@<.5ex>@/_/[d] \ar@<-.5ex>@/_/[d] 
& B \ar@<.5ex>@/^/[d] \ar@<-.5ex>@/^/[d]\\
C \ar@<.5ex>@/_/[r] \ar@<-.5ex>@/_/[r] & D}$$

\medskip

\begin{verbatim}
$$\xymatrix{A \ar@<.5ex>@/^/[r] \ar@<-.5ex>@/^/[r] 
\ar@<.5ex>@/_/[d] \ar@<-.5ex>@/_/[d] 
& B \ar@<.5ex>@/^/[d] \ar@<-.5ex>@/^/[d]\\
C \ar@<.5ex>@/_/[r] \ar@<-.5ex>@/_/[r] & D}$$
\end{verbatim}

\end{itemize}

\end{frame}